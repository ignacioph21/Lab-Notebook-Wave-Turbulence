\entry{Antecedentes Teóricos}


\section{Formalismo Hamiltoniano para ondas en medios continuos.}
\subsection*{Las bases.}
En el caso más simple posible podemos describir un medio continuo a partir de un par de variables canónicas $q(\vec r, t)$ y $p(\vec r, t)$ de forma tal que las ecuaciones de movimiento serán:

\begin{equation}
	\frac{\partial q}{\partial t} = \frac{\delta \mathcal{H}}{\delta p} \qquad \frac{\partial p}{\partial t} = -\frac{\delta \mathcal{H}}{\delta q}
	\label{eq:Hamilton_continuum}
\end{equation}

Aquí tenemos la densidad Hamiltoniana $\mathcal{H}$, y los $\delta$ representan la \textit{derivada funcional}, análoga a la derivada parcial en el límite continuo. Vamos a tener las propiedades análogas a la versión para variables finitas:

\begin{equation}
	\frac{\delta q(\vec r)}{\delta q(\pvec r')} = \delta^3(\vec r - \pvec r') \qquad \frac{\delta p(\vec r)}{\delta p(\pvec r')} = \delta^3(\vec r - \pvec r')
\end{equation}

Además, aplica la regla de la cadena:

\begin{equation}
	\frac{\delta f(\vec r)}{\delta q(\pvec r')} = \frac{\partial f(\vec r)}{\partial q(\vec r)} \frac{\delta q(\vec r)}{\delta q(\pvec r')} \qquad \frac{\delta f(\vec r)}{\delta p(\pvec r')} = \frac{\partial f(\vec r)}{\partial p(\vec r)} \frac{\delta p(\vec r)}{\delta p(\pvec r')}
\end{equation}

Y conmuta con derivadas parciales de las variables espaciales (e.g. $\partial_{r^i}$) y con integración en variales especiales (e.g. $\int d^3r$).


Además será importante definir el \textbf{corchete de Poisson}. Para un conjunto de variables discretas $(q_i,p_i)$ es (el orden de los dos términos es convención):

\begin{equation}
	\{A, B\} = \sum_i\left(\frac{\partial A}{\partial p_i}\frac{\partial B}{\partial q_i} - \frac{\partial A}{\partial q_i}\frac{\partial B}{\partial p_i}\right)
\end{equation}

De forma que en variables canónicas tenemos:

\begin{equation}
	\{q_i,q_j\}=\{p_i,p_j\}=0 \qquad \{q_i,p_j\} = \delta_{ij}
\end{equation}

Y en el límite continuo esto resulta en:

\begin{equation}
	\{A, B\} = \int d^3r \left(\frac{\delta A}{\delta p}\frac{\delta B}{\delta q} - \frac{\delta A}{\delta q}\frac{\delta B}{\delta p}\right)
\end{equation}

Y la condición de canonicidad queda expresada como:

\begin{equation}
	\{q(\vec r),q(\pvec r')\}=\{p(\vec r),p(\pvec r)\}=0 \qquad \{q(\vec r),p(\pvec r')\} = \delta^3(\vec r- \pvec r')
\end{equation}

Esto es fundamental, ya que las transformaciones canónicas, que preservan la forma de las ecuaciones \eqref{eq:Hamilton_continuum}, serán aquellas que preservan estos corchetes de Poisson ($\{A(q',p'), B(q',p')\}_{q,p}$).


\subsection*{Teoremas de Conservación y simetrías.}
La derivada temporal de una función (con una posible dependencia explícita en $t$) será:

\begin{equation}
	\frac{df(q,p,t)}{dt} = {H, f} + \frac{\partial f}{\partial t}
\end{equation}

En general no hay dependencias explícitas, con lo cual tenemos que si ${H, f}=0$ entonces la cantidad $f$ es constante de movimiento.



\subsection*{Pasar a variables complejas.}
Ahora bien, se puede mostrar la siguiente transformación es canónica (a menos de un factor multiplicativo):

\begin{equation}
	a(\vec r,t)=\frac{1}{\sqrt{2}}(Q(\vec r,t)+iP(\vec r,t)) \qquad a^*(\vec r,t)=\frac{1}{\sqrt{2}}(Q(\vec r,t)-iP(\vec r,t))
\end{equation}

Con $Q=\lambda q$ y $P=p / \lambda$ tales que $Q$ y $P$ tengan las mismas unidades. La ventaja de pasar a las variables complejas es que se reduce de las dos ecuaciones \eqref{eq:Hamilton_continuum} a una sola (la segunda es la conjugada de la primera):

\begin{equation}
	i\frac{\partial a}{\partial t} = \frac{\delta \mathcal{H}}{\delta a^*}
\end{equation}

Con los corchetes en las coordenadas originales\footnote{A veces, parece que en las nuevas variables ($\{\cdot,\cdot\}_{a,a^*}$) se escribe el corchete con un factor $i$ adelante}:

\begin{equation}
	\{a(\vec r), a(\pvec r')\}_{Q,P} = \{a^*(\vec r), a^*(\pvec r')\}_{Q,P} = 0 \qquad \{a(\vec r), a^*(\pvec r')\}_{Q,P} = i\delta(\vec r - \pvec r')
\end{equation}

Estas variables complejas son análogas a los operadores de subida y bajada en cuántica. 

\subsection*{Las cantidades conservadas}
En un primer lugar tendremos la conservación de la energía total (siempre que trabajemos con el modelo más sencillo), pero en el caso que el Hamiltoniano tenga simetría $U(1)$ en las variables $a$, habrá otra ley de conservación asociada $N=|a|^2$, que es análogo al número de partículas en el caso cuántico. Ya desarrollaremos esto más adelante. 

\subsection*{Paso al espacio de Fourier.}
Vamos a imaginar que estamos en una caja de largo $L$ con condiciones de contorno periódicas, que eventualmente podemos hacer tender a infinito, y vamos a expresar a la variable $a(\vec r)$ como una combinación de las variables $a_{\vec k} \equiv a(\vec k)$ tal que:

\begin{equation}
	a_{\vec k} = a(\vec k) = \frac{1}{V}\int d\vec r\;a(\vec r)e^{-i\vec k\cdot\vec r} \qquad a(\vec r) = \sum_{\vec k} a_{\vec k} e^{i\vec k\cdot\vec r}
\end{equation}

Es posible mostrar usando los corchetes para la variable $a$ en el espacio real que las nuevas variables en el espacio de momento $a_{\vec k}$ satisfacen los mismos corchetes de Poisson, a menos de un factor de volumen. Entonces las ecuaciones de movimiento mantienen su forma:

\begin{equation}
	i\frac{\partial a_{\vec k}}{\partial t} = \frac{\delta \mathcal{H}}{\delta a^*}
\end{equation}

\section{Desarrollo perturbativo del Hamiltoniano.}
Nos va a interesar escribir el Hamiltoniano como una serie de potencias de $a$ y $a^*$. El término de orden 0, i.e. $\mathcal{H}_0$, no es relevante, ya que es una constante que no aparece en la ecuación de movimiento al derivar, y el término de orden 1, $\mathcal{H}_1$, podemos eliminarlo asumiendo que el medio está en equilibrio cuando las amplitudes de las ondas son 0, y el mínimo está en $a=a^*=0$. De esta forma nos queda:

\begin{equation}
	\mathcal{H} = \mathcal{H}_2 + \mathcal{H}_3 + \mathcal{H}_4 + \cdots \equiv \mathcal{H}_2 + \mathcal{H}_{int}
\end{equation}

\subsection*{El orden más bajo.}

Donde $\mathcal{H}_{int}$ representa las interacciones entre ondas. Se puede mostrar que existe una transformación canónica a unas variables $b_{\vec{k}},b_{\vec{k}^*}$ tales que:

\begin{equation}
	\mathcal{H}_2 = \int \omega(\vec k)b_{\vec k}b^*_{\vec k}
\end{equation} 

De forma tal que a primer orden no nulo la ecuación de movimiento resulta trivial:

\begin{equation}
	\frac{\partial b_{\vec{k}}}{\partial t}=-i\omega b_{\vec{k}} \qquad \Rightarrow{} \qquad b_{\vec{k}}=b(\vec k, 0)e^{-i\omega t}
\end{equation}

\subsection*{El Hamiltoniano de interacción.}






\section{Espectro de energía y PSD.}
Nos interesará estudiar la transferencia de energía entre escalas experimentalmente, que estará caracterizada por la densidad espectral de energía $E_k$ tal que la magnitud $E = \int E_k dk$ se conserva. 

Para esto vamos a empezar dando las definiciones básicas de lo que estaremos usando.

Como mencioné anteriormente, la energía total va a venir dada por una distribución, que puede escribirse ya sea en el espacio de $k$ o de $\omega$, vinculados ambos a través de la relación de dispersión lineal $\omega(k)$. Entonces:

\begin{equation}
	E = \int E(\vec k) d\vec k = \int k E(\vec k) dkd\theta \equiv \int E(k) dk = \int E(\omega) d\omega
\end{equation}

O sea, estamos definiendo $E(k) = 2\pi E(\vec k)$ y la relación entre las densidades en espacio de frecuencia y momento está dada por: $E(k) dk = E(\omega) d\omega$.

Esta energía será la de la superficie libre. En principio sabemos que se dará un balance entre la energía cinética y la potencial, como una especie de equipartición \cite{kunduFluidMechanics2014}. Se suele calcular la potencial (a partir de las mediciones para la superficie libre $\eta$) y se multiplica por 2 \cite{deikeEtudesExperimentalesNumeriques2013}. Vamos a entonces trabajar con las energías potenciales, cuya expresión depende de si tenemos ondas de gravedad o capilares. Tenemos que:

\begin{equation}
	E_p^g = \frac{1}{2} \rho g \int \eta^2 dS \qquad E_p^s = \gamma \int\sqrt{1+(\nabla\eta)^2} - 1 dS
\end{equation}

Acá la energía por capilaridad es básicamente la integral del elemento de superficie $ds$. Se puede aproximar a primer orden para una onda armónica en el caso lineal (con $eta\ll1$), entonces \cite{deikeEtudesExperimentalesNumeriques2013}

\begin{equation}
	E_p^s = \frac{1}{2} \gamma\int k^2\eta^2dS
\end{equation}

Usando que $(1+\varepsilon)^n\approx1+n\varepsilon$. Ahora vamos a querer pasar al espacio de Fourier, transformando las energías:

\begin{equation}
	E_p^g = \frac{\rho g}{2} \int |\eta_{\vec k}|^2d\vec k \qquad E_p^s = \frac{\gamma}{2} \int k^2|\eta_{\vec k}|^2d\vec k
\end{equation}

Donde $\eta_{\vec k}$ es la transformada de Fourier de $\eta$:

\begin{equation}
	\eta(\vec k) = \frac{1}{2\pi}\int\eta(\vec r)e^{-i\vec k\cdot\vec r} d\vec r
\end{equation}

Podemos integrar en cilíndricas para definir $\eta_k = 2\pi k\eta_{\vec k}$. Las densidades espectrales de energía van a ser los integrandos. 

Por último vamos a definir la Power Spectral Density (PSD) que va a ser lo más sencillo para trabajar experimentalmente:

\begin{equation}
	S_\eta(\omega) = \frac{1}{T} |\eta_\omega|^2 \qquad S_\eta(k) = \frac{1}{L^2} |\eta_k|^2
\end{equation}

Acá ahora $\eta(\omega) = \frac{1}{\sqrt{2\pi}}\int\eta(t)e^{i\omega t}$ es la transformada de Fourier respecto al tiempo. Además para obtener ya sea $\eta(t)$ o $\eta(\vec r)$ se puede promediar en el tiempo o espacio, con lo cual en realidad sería $S_\eta\propto\langle |\eta|^2 \rangle$ (y técnicamente el promedio sería en el ensamble, pero consideramos al sistema ergódico). \cite{falconExperimentsSurfaceGravity2022}

Podemos relacionar las PSD en ambos espacios como $S(\omega)d\omega=S(k)dk$. De esta forma entonces podemos obtener la densidad de energía a partir de la PSD como:

\begin{equation}
		E^g(k) = \frac{\rho g}{2} S_\eta(k) \qquad E^s(k)=\frac{\gamma}{2}k^2S_\eta(k)
\end{equation}


Una cosa interesante es que si escribimos explícitamente $S_\eta(k) \propto k|\eta_{\vec k}|^2$ entonces nos queda en ambos casos:

\begin{equation}
	E(k) \propto \omega^2|\eta_{\vec k}|^2 \propto |\dot \eta_{\vec k}|^2 
\end{equation}

Que sería la energía cinética efectivamente.


\section{Interacciones Resonantes de $N$ ondas} % 








\section{Espectro de Kolmogorov-Zakharov-Filonenko.}
Experimentalmente es posible observar que el espectro de energía para la elevación de la superficie libre en un determinado rengo de números de ondas (de equilibrio) resulta $E_\omega  \sim A\omega^s$. En este rango la mayor contribución a la dinámica se debe a los efectos no lineales, ya que los viscosos son menores. Es posible calcular los exponentes de forma analítica a partir de las ecuaciones para la superficie libre \cite{zakharovEnergySpectrumStochastic1967}

La densidad de energía se puede escribir como:

\begin{equation}
	E(\vec k) = \omega(\vec k) n(\vec k)
\end{equation}  

Donde $n$ es la \textit{wave action}, análoga a la cantidad de partículas (o ondas) con ese número de onda.

\subsection*{Ondas de gravedad \cite{zakharovEnergySpectrumStochastic1967}}
Tenemos que $E(\omega)=\omega^4n(\omega)$ y pueden calcular que para $n(\omega)=A\omega^s$ hay soluciones $s=-1$, que se corresponde con la distribución de Rayleigh-Jeans (equilibrio termodinámico para el caso lineal, con PDF Gaussiana \cite{nazarenkoWaveTurbulence2011}), y $s=-8$ que es la análoga al espectro de Kolmogorov para turbulencia hidrodinámica.

Las equivalencias son \cite{falconExperimentsSurfaceGravity2022}:

\begin{equation}
	E_k^g \sim P^{1/3} g^{1/2} k^{-5/2} \qquad S_k^g \sim P^{1/3} g^{-1/2} k^{-5/2} \qquad S_\omega^g  \sim P^{1/3} g \omega^{-4}
\end{equation}  

Es posible llegar a estos resultados de forma dimensional, pero para que el exponente sea único hay que asumir a priori que el flujo en el espectro de energía va en el orden $P^{1/(N-1)}$ con $N$ el número de ondas interactuantes más bajo, siendo $N=4$ para ondas de gravedad.

Para hacer los cambios hay que usar la relación de dispersión:

\begin{equation}
	\omega^2=gk
\end{equation}

\subsection*{Ondas de capilaridad \cite{zakharovWeakTurbulenceCapillary1971}}
Al igual que para las ondas de gravedad calculan $n(\omega)$, llegando esta vez a los resultados de $s=-1$ que es la de Rayleigh-Jeans y $s=-17/6$ que es la que nos interesa. Con esto:

\begin{equation}
	E_k^s \sim P^{1/2} \left(\frac{\gamma}{\rho}\right)^{1/4} k^{-7/4} \qquad S_k^s \sim P^{1/2} \left(\frac{\gamma}{\rho}\right)^{-3/4} k^{-15/4} \qquad S_\omega^s  \sim P^{1/2} \left(\frac{\gamma}{\rho}\right)^{1/6} \omega^{-17/6}
\end{equation}   

Acá la relación de dispersión es:

\begin{equation}
	\omega^2 = \frac{\gamma}{\rho}k^3
\end{equation}



\subsection{Doble cascada para $N$ par} % a 
A partir de los resultados para los espectros KZF se puede ver que existe una cascada directa de energía, sin embargo, para el caso en que las interacciones se dan entre un número par de ondas, tenemos otro invariante además de la energía, que es la \textit{wave action}, que tendrá una cascada inversa. \cite{nazarenkoWaveTurbulence2011}

Para el caso de ondas de gravedad, con interacciones de 4 ondas, tenemos el espectro:

\begin{equation}
	E^g_k \sim g^{2/3} \zeta^{1/3} k^{-7/3}
\end{equation}

Con $\zeta$ el flujo de \textit{wave action} (ratio de disipación)

%Estos son los antecedentes Teóricos. Más antecedentes.
%
%Esta es una ecuación:
%
%\begin{equation}
%	A = B + \phi
%	\label{eq:eq_1}
%\end{equation}
%
%Y esta es u referencia a la ecuación \eqref{eq:eq_1}. 
%
%Y por último una prueba del auto-guardado en Github. Y una más. 
%Y esta sería la última. 

% IMÁGENES Y TABLAS
%
% A la izquierda 
% \begin{figure}[!ht]
%	     \begin{minipage}[c]{0.5\textwidth}
%		         \centerfloat
%		         \includegraphics[width=0.8\textwidth]{}
%		     \end{minipage}
%	         \begin{minipage}[c]{0.49\textwidth}
%		         \captionsetup{width=\textwidth}  
%		         \caption{.}
%		     \label{fig:}
%		     \end{minipage}
%	 \end{figure}
%
%A la derecha
% \begin{figure}[!ht]
%	     \begin{minipage}[c]{0.5\textwidth}
%		           \captionsetup{width=\textwidth}  
%		           \caption{.}
%		           \label{fig:}
%		     \end{minipage}
%	     \begin{minipage}[c]{0.49\textwidth}
%		         \centerfloat
%		         \includegraphics[width=0.8\textwidth]{}
%		     \end{minipage}
%	 \end{figure}
% 
% Ancho completo
% \begin{figure}[!ht]
%	     \centerfloat
%	     \includegraphics[width=0.9\textwidth]{}
%	     \caption{}
%	     \label{fig:}
%	 \end{figure}
%
% Wrapfigure
% \begin{wrapfigure}{r}{0.5\textwidth}
%	         \vspace{-20pt}
%	         \centerfloat
%	         \includesvg[width=0.3\textwidth]{}
%	         \captionsetup{width=0.45\textwidth}
%	         \caption{.}
%	         \label{fig:}
%	 \end{wrapfigure}
%
% Generador de tabals: https://www.tablesgenerator.com/
% Tablas con líneas más gruesas: https://tex.stackexchange.com/questions/41758/how-can-i-reproduce-this-table-with-thick-lines
% 
% Tabla y figura (https://tex.stackexchange.com/questions/417505/table-just-below-a-figure)
% \begin{minipage}{0.99\textwidth}
%	   \begin{minipage}[c]{0.49\textwidth}
%		         \centerfloat
%		         \includegraphics[width=\textwidth]{}
%		         \centering
%		         \captionsetup{width=.8\linewidth}
%		         \captionof{figure}{.}
%		         \label{fig:}
%		   \end{minipage}
%	   \hfill
%	   \begin{minipage}[c]{0.49\textwidth}
%		         \centerfloat
%		         \includegraphics[width=\textwidth]{}
%		         \centering
%		         \captionsetup{width=.8\linewidth}
%		         \captionof{table}{.}
%		          \label{tab:}
%		     \end{minipage}
%	 \end{minipage}
% 
% Subfiguras
%\begin{figure}[!ht]
%	\begin{minipage}[c]{0.5\textwidth}
%		\begin{subfigure}{\textwidth}
%			\centering
%			\includegraphics[width=0.8\textwidth]{}
%			\captionsetup{width=0.8\textwidth}
%			\subcaption{.}
%			\label{fig:}
%		\end{subfigure}
%	\end{minipage}\begin{minipage}[c]{0.49\textwidth}
%		\begin{subfigure}{\textwidth}
%			\centering
%			\includegraphics[width=0.8\textwidth]{}
%			\captionsetup{width=0.8\textwidth}
%			\subcaption{.}
%			\label{fig:}
%		\end{subfigure}
%	\end{minipage}
%	\caption{}
%	\label{fig:}
%\end{figure}
% 
% Tablas/figuras iguales side by side
% \begin{table}[]
%	     \parbox{0.5\textwidth}{
%		     \centering
%		     \includegraphics[width=0.95\linewidth]{}
%		     \caption{}
%		     \label{tab:}
%		 }
%	 \parbox{0.5\textwidth}{
%		     \centering
%		     \includegraphics[width=0.9\linewidth]{}
%		     \caption{}
%		     \label{tab:}
%		 }
%	 \end{table}
% 
% Tablas ancho completo
% \begin{table}[!h]
%	     \centerfloat
%	     \caption{.}
%	     \label{tab:}
%	 \end{table}