\entry{Antecedentes Teóricos}


\section{Introducción General.}
\subsection*{Turbulencia}
La turbulencia es un estado de un sistema físico no lineal fuera del equilibrio que cuenta con una distribución de energía sobre muchos grados de libertad. Puede mantenerse mediante un forzado, que inyecta energía, o decaer hasta el equilibrio. \cite{falkovichTurbulence2006}

Esto ocurre para números de Reynold suficientemente grandes:

\begin{equation}
	\text{Re} \equiv \frac{UL}{\nu}
\end{equation}

Las perturbaciones producidas para una escala L tienen un bajo efecto disipativo a causa de la viscosidad respecto de los efectos no lineales, que serán los dominantes. Estas no linealidades inducen movimiento en escalas cada vez más chicas, hasta llegar a la escala disipativa donde la viscosidad lidera, en un rango mucho menor a L. En el intermedio existe un amplio rango (llamado inercial) donde la viscosidad es despreciable y domina la no linealidad. 

El flujo de energía hacia las escalas más chicas se da en un proceso de tipo \textit{cascada}. Esta idea fue formulada por Richardson (1922):

\begin{center}
	\textit{Big whorls have little whorls,} \\
	\textit{Which feed on their velocity,} \\
	\textit{And little whorls have lesser whorls,} \\ 
	\textit{And so on to viscosity.}
\end{center}

Donde por la ruptura de vórtices grandes se forman más y más chicos. Por esta idea también a veces se le llama a esta turbulencia \textit{Eddy Turbulence}. Además asumimos que la cascada es \textit{local}, o sea que se produce una transferencia de energía de forma continua entre escalas.

\begin{figure}[!th]
	\centering
	\includegraphics[width=0.53\linewidth]{Figures/Antecedentes_teoricos/Energy_cascade}
	\caption{Representación pictórica de la cascada de energía de Richardson. Imagen tomada de \cite{nazarenkoWaveTurbulence2011}.}
	\label{fig:energycascade}
\end{figure}


Para poder describir a un sistema dinámico tan complejo es necesario recurrir a herramientas estadísticas. 


Kolmogorov-Obukhov plantearon un argumento dimensional donde los únicos parámetros relevantes en el rango inercial son el flujo de energía $\varepsilon$ y $k$. En particular para el caso de turbulencia homogénea (no depende de la posición) e isótropa (espectro no depende de la dirección de $\vec k$), o HIT, el espectro de energía en Fourier resulta:

\begin{equation}
	E(k) = 4\pi k^2E(\vec k) = C\varepsilon^{2/3}k^{-5/3}
	\label{eq:Kolmogorov_5_3}
\end{equation}

Que es uno de los resultados más fuertes sobre turbulencia, en acuerdo con muchos resultados experimentales. Sin embargo, para momentos más altos del campo de velocidades empieza a haber diferencias más y más grandes, debido al fenómeno de \textit{intermitencia}, donde se producen fluctuaciones en locales en $\varepsilon$.

\subsection*{Turbulencia 2D}
Cuando tenemos un flujo ideal incompresible en 2D tenemos dos cantidades conservadas:

\begin{equation}
	E = \frac{1}{2}\langle \vec u^2 \rangle = \int E(k) dk \qquad \Omega = \frac{1}{2} \langle \vec \omega^2 \rangle = \int k^2 E(k) dk
\end{equation}

Que son la energía y enstrofía por unidad de área. Vamos a tener entonces una \textit{cascada dual}, de energía y enstrofía. Para saber de qué tipo va a ser cada una recurrimos al argumento de Fjørtoft:

Supongamos que inyectamos energía a un ratio $\varepsilon$ en la escala $k_0$ y por tanto enstrofía a $\eta\sim k_0^2\varepsilon$. Imaginemos que disipamos energía en $k_+\ll k_0$ como ocurriría en el caso 3D, a un ritmo equivalente al de inyección $\varepsilon$. Entonces la enstrofía se disiparía a un ritmo $k_+^2\varepsilon\gg k_0^2\varepsilon\sim \eta$. Esto es una contradicción ya que se estaría disipando enstrofía a un ritmo mayor que el que se inyecta, lo cual no es posible en un sistema estadístico estacionario. Por tanto debe ser que la energía se disipa en $k_-\ll k_0$, y entonces es una \textit{cascada inversa de energía}. Por una lógica análoga concluimos que tenemos una \textit{cascada directa de enstrofía}. De esta forma los vórtices pequeños son estirados por los grandes, y se van uniendo.



\begin{figure}[!ht]
	\centering
	\includegraphics[width=0.7\linewidth]{Figures/Antecedentes_teoricos/Dual_Energy_cascade}
	\caption{Cascada dual de energía y enstrofía en turbulencia 2D. Imagen de \cite{nazarenkoWaveTurbulence2011}.}
	\label{fig:dualenergycascade}
\end{figure}


El espectro de energía en la cascada inversa sigue la misma ley de escala anterior, pero con otra constante, mientras que la ley de escala para la cascada directa resulta de tomar como parámetro a $\eta$ en lugar de $\varepsilon$:

\begin{equation}
	E(k) = C_\eta \eta^{2/3} k^{-3} % .
\end{equation}



\subsection*{Turbulencia de Ondas}
La turbulencia de ondas es una descripción estadística de un sistema mecánico fuera del equilibrio formado por ondas no lineales aleatorias. La diferencia principal respecto a la Turbulencia clásica es que ahora vamos a tener un sistema de ondas en lugar de un sistema de vórtices. 

Una ventaja importante de este tipo de turbulencia es que surge una clausura de forma natural para la jerarquía de momentos estadísticos, que permite calcular más resultados de forma analítica.

\section{Formalismo Hamiltoniano para ondas en medios continuos.}
\subsection*{Las bases.}
En el caso más simple posible podemos describir un medio continuo a partir de un par de variables canónicas $q(\vec r, t)$ y $p(\vec r, t)$ de forma tal que las ecuaciones de movimiento serán:

\begin{equation}
	\frac{\partial q}{\partial t} = \frac{\delta \mathcal{H}}{\delta p} \qquad \frac{\partial p}{\partial t} = -\frac{\delta \mathcal{H}}{\delta q}
	\label{eq:Hamilton_continuum}
\end{equation}

Aquí tenemos la densidad Hamiltoniana $\mathcal{H}$, y los $\delta$ representan la \textit{derivada funcional}, análoga a la derivada parcial en el límite continuo. Vamos a tener las propiedades análogas a la versión para variables finitas:

\begin{equation}
	\frac{\delta q(\vec r)}{\delta q(\pvec r')} = \delta^3(\vec r - \pvec r') \qquad \frac{\delta p(\vec r)}{\delta p(\pvec r')} = \delta^3(\vec r - \pvec r')
\end{equation}

Además, aplica la regla de la cadena:

\begin{equation}
	\frac{\delta f(\vec r)}{\delta q(\pvec r')} = \frac{\partial f(\vec r)}{\partial q(\vec r)} \frac{\delta q(\vec r)}{\delta q(\pvec r')} \qquad \frac{\delta f(\vec r)}{\delta p(\pvec r')} = \frac{\partial f(\vec r)}{\partial p(\vec r)} \frac{\delta p(\vec r)}{\delta p(\pvec r')}
\end{equation}

Y conmuta con derivadas parciales de las variables espaciales (e.g. $\partial_{r^i}$) y con integración en variales especiales (e.g. $\int d^3r$).


Además será importante definir el \textbf{corchete de Poisson}. Para un conjunto de variables discretas $(q_i,p_i)$ es (el orden de los dos términos es convención):

\begin{equation}
	\{A, B\} = \sum_i\left(\frac{\partial A}{\partial p_i}\frac{\partial B}{\partial q_i} - \frac{\partial A}{\partial q_i}\frac{\partial B}{\partial p_i}\right)
\end{equation}

De forma que en variables canónicas tenemos:

\begin{equation}
	\{q_i,q_j\}=\{p_i,p_j\}=0 \qquad \{q_i,p_j\} = \delta_{ij}
\end{equation}

Y en el límite continuo esto resulta en:

\begin{equation}
	\{A, B\} = \int d^3r \left(\frac{\delta A}{\delta p}\frac{\delta B}{\delta q} - \frac{\delta A}{\delta q}\frac{\delta B}{\delta p}\right)
\end{equation}

Y la condición de canonicidad queda expresada como:

\begin{equation}
	\{q(\vec r),q(\pvec r')\}=\{p(\vec r),p(\pvec r)\}=0 \qquad \{q(\vec r),p(\pvec r')\} = \delta^3(\vec r- \pvec r')
\end{equation}

Esto es fundamental, ya que las transformaciones canónicas, que preservan la forma de las ecuaciones \eqref{eq:Hamilton_continuum}, serán aquellas que preservan estos corchetes de Poisson ($\{A(q',p'), B(q',p')\}_{q,p}$).


\subsection*{Teoremas de Conservación y simetrías.}
La derivada temporal de una función (con una posible dependencia explícita en $t$) será:

\begin{equation}
	\frac{df(q,p,t)}{dt} = \{H, f\} + \frac{\partial f}{\partial t}
\end{equation}

En general no hay dependencias explícitas, con lo cual tenemos que si $\{H, f\}=0$ entonces la cantidad $f$ es constante de movimiento.



\subsection*{Pasar a variables complejas.}
Ahora bien, se puede mostrar que la siguiente transformación es canónica (a menos de un factor multiplicativo):

\begin{equation}
	a(\vec r,t)=\frac{1}{\sqrt{2}}\big(Q(\vec r,t)+iP(\vec r,t)\big) \qquad a^*(\vec r,t)=\frac{1}{\sqrt{2}}\big(Q(\vec r,t)-iP(\vec r,t)\big)
\end{equation}

Con $Q=\lambda q$ y $P=p / \lambda$ tales que $Q$ y $P$ tengan las mismas unidades. La ventaja de pasar a las variables complejas es que se reduce de las dos ecuaciones \eqref{eq:Hamilton_continuum} a una sola (la segunda es la conjugada de la primera):

\begin{equation}
	i\frac{\partial a}{\partial t} = \frac{\delta \mathcal{H}}{\delta a^*}
\end{equation}

Con los corchetes en las coordenadas originales\footnote{A veces, parece que en las nuevas variables ($\{\cdot,\cdot\}_{a,a^*}$) se escribe el corchete con un factor $i$ adelante.}:

\begin{equation}
	\{a(\vec r), a(\pvec r')\}_{Q,P} = \{a^*(\vec r), a^*(\pvec r')\}_{Q,P} = 0 \qquad \{a(\vec r), a^*(\pvec r')\}_{Q,P} = i\delta(\vec r - \pvec r')
\end{equation}

Estas variables complejas son análogas a los operadores de subida y bajada en cuántica. 

\subsection*{Las cantidades conservadas}
En un primer lugar tendremos la conservación de la energía total (siempre que trabajemos con el modelo más sencillo), pero en el caso que el Hamiltoniano tenga simetría $U(1)$ en las variables $a$, habrá otra ley de conservación asociada, $N=|a|^2$, que es análogo al número de partículas en el caso cuántico. Ya desarrollaremos esto más adelante. 

\subsection*{Paso al espacio de Fourier.}
Vamos a imaginar que estamos en una caja de largo $L$ con condiciones de contorno periódicas, que eventualmente podemos hacer tender a infinito, y vamos a expresar a la variable $a(\vec r)$ como una combinación de las variables $a_{\vec k} \equiv a(\vec k)$ tal que:

\begin{equation}
	a_{\vec k} = a(\vec k) = \frac{1}{V}\int d\vec r\;a(\vec r)e^{-i\vec k\cdot\vec r} \qquad a(\vec r) = \sum_{\vec k} a_{\vec k} e^{i\vec k\cdot\vec r}
\end{equation}

Es posible mostrar usando los corchetes para la variable $a$ en el espacio real que las nuevas variables en el espacio de momento $a_{\vec k}$ satisfacen los mismos corchetes de Poisson, a menos de un factor de volumen. Entonces las ecuaciones de movimiento mantienen su forma:

\begin{equation}
	i\frac{\partial a_{\vec k}}{\partial t} = \frac{\delta \mathcal{H}}{\delta a_{\vec k}^*}
\end{equation}


\section{Desarrollo perturbativo del Hamiltoniano.}
Nos va a interesar escribir el Hamiltoniano como una serie de potencias de $a$ y $a^*$. El término de orden 0, i.e. $\mathcal{H}_0$, no es relevante, ya que es una constante que no aparece en la ecuación de movimiento al derivar, y el término de orden 1, $\mathcal{H}_1$, podemos eliminarlo asumiendo que el medio está en equilibrio cuando las amplitudes de las ondas son 0, y el mínimo está en $a=a^*=0$. De esta forma nos queda:

\begin{equation}
	\mathcal{H} = \mathcal{H}_2 + \mathcal{H}_3 + \mathcal{H}_4 + \cdots \equiv \mathcal{H}_2 + \mathcal{H}_{int}
\end{equation}

Donde $\mathcal{H}_{int}$ representa las interacciones entre ondas. 
\subsection*{El orden más bajo.}

Se puede mostrar que existe una transformación canónica a unas nuevas variables $b_{\vec{k}},b_{\vec{k}^*}$ tales que \cite{zakharovKolmogorovSpectraTurbulence1992}:

\begin{equation}
	\mathcal{H}_2 = \int \omega(\vec k)b_{\vec k}b^*_{\vec k} d\vec k 
\end{equation} 

De forma tal que a primer orden no nulo la ecuación de movimiento resulta trivial:

\begin{equation}
	\frac{\partial b_{\vec{k}}}{\partial t}=-i\omega b_{\vec{k}} \qquad \Rightarrow{} \qquad b_{\vec{k}}=b(\vec k, 0)e^{-i\omega t}
\end{equation}

En este punto la única diferencia entre distintos medios radica en la relación de dispersión $\omega(\vec k)$.

\subsection*{El Hamiltoniano de interacción.}
Los términos de orden superior se pueden interpretar como términos de interacción entre ondas, en analogía a lo que ocurre en mecánica cuántica. Vamos a usar la notación reducida: $b_{\vec k_1} \equiv b_1$.

Para las interacciones de tres ondas tenemos:

\begin{equation}
	\begin{split}
		\mathcal{H}_3 &= \frac{1}{2}\int\big(V_{123}b^*_1b_2b_3+\text{c.c.}\big)\delta(\vec k_1-\vec k_2 -\vec k_3) d\vec k_1d\vec k_2d\vec k_3 \\ &+ \frac{1}{6} \int\big(U_{123}b^*_1b^*_2b^*_3+\text{c.c.}\big)\delta(\vec k_1 + \vec k_2 + \vec k_3) d\vec k_1d\vec k_2d\vec k_3
	\end{split}
\end{equation} 

Donde tenemos unos coeficientes $V_{123}$ y $U_{123}$ que nos hablan de la magnitud de la interacción. El primer término describe procesos de la forma $1\rightarrow2$ y  $2\rightarrow1$, mientras que el segundo describe procesos del tipo $3\rightarrow0$ y $0\rightarrow3$, que son aniquilaciones de tres ondas hacia el vacío o creación espontánea de tres ondas por fluctuaciones. % 

Por otro lado, para las interacciones de cuatro ondas tenemos:

\begin{equation}
	\begin{split}
		\mathcal{H}_4 &= \frac{1}{4} \int W_{1234}b1^*b_2^*b_3b_4\delta(\vec k_1+\vec k_2-\vec k_3-\vec k_4) d\vec k_1d\vec k_2d\vec k_3d\vec k_4 \\
		&+\int \big(G_{1234}b_1b_2^*b_3^*b_4^* + \text{c.c.}\big) \delta(\vec k_1-\vec k_2-\vec k_3-\vec k_4) d\vec k_1d\vec k_2d\vec k_3d\vec k_4 \\
		&+\int \big(R^*_{1234}b_1b_2b_3b_4 + \text{c.c.}\big) \delta(\vec k_1+\vec k_2+\vec k_3+\vec k_4) d\vec k_1d\vec k_2d\vec k_3d\vec k_4
	\end{split}	
\end{equation}

Donde ahora tenemos procesos del tipo $2\rightarrow2$ para el primer término, $1\rightarrow3$ y $3\rightarrow1$ para el segundo, $4\rightarrow0$ y $0\rightarrow4$ para el tercero.
 
Los coeficientes de interacción van a tener las simetrías asociadas a la interacción, o sea, se van a poder intercambiar los índices asociados a "partículas" de cada lado de la flecha, por ejemplo, $V_{123}=V_{132}$.

Notemos además que si hay presentes términos con una cantidad distinta de "partículas" de un lado que del otro entonces no habrá simetría $U(1)$ y no se conservará la cantidad de partículas asociadas a ese proceso. 


\subsection*{Interacciones Resonantes de $N$ ondas} % 
En general vamos a poder tener interacciones de tanta cantidad de ondas como querramos, a medida que aumentemos el orden del Hamiltoniano de interacción. Las derivadas de estos términos del Hamiltoniano son integrales de colisiones $I_{\vec k}^N$ de $N$ ondas, como las de la Ecuación de Boltzmann. 

Ahora bien, en general va a ser suficiente con llegar al primer término no nulo, tal que se satisfagan la conservación de lo que en cuántica serían el momento y la energía: 

\begin{equation}
	\vec k_1 \pm \vec k_2 \pm \cdots \pm \vec k_N = 0 \qquad \omega(\vec k_1) \pm \omega(\vec k_2) \pm \cdots \pm \omega(\vec k_N) = 0
	\label{eq:condición_resonancia}
\end{equation}

Los signos corresponderán a si la "partícula'' ingresa o sale. 

Si no se satisface la segunda condición de resonancia entonces el término correspondiente del Hamiltoniano puede eliminarse mediante transformaciones canónicas apropiadas, como se verá más adelante \cite{zakharovKolmogorovSpectraTurbulence1992}.

Para el caso en que tenemos una relación de dispersión del tipo $\omega\sim k^{\alpha}$ la condición de resonancia de tres ondas no será posible en los casos en que $\alpha<1$, con lo cual el primer término resonante será el de cuatro ondas, para el cual siempre hay solución en el caso de $\omega\sim k^{\alpha}$. La demostración se puede entender gráficamente, si dibujamos las superficies en 2D para $\omega(\vec k, \vec k_0)$ y $\omega(\vec k) + \omega(\vec k_0)$, donde $\vec k_0$ es un parámetro que puede variarse. Si las curvas se intersecan hay solución, sino solo existe la solución trivial en que ambas curvas son iguales, para el caso en que una de las ondas es 0. Esto está ilustrado en la Figura \ref{fig:solución_condición_resonante}. % STO 

\begin{figure}[!ht]
	\begin{minipage}[c]{0.5\textwidth}
		\begin{subfigure}{\textwidth}
			\centering
			\includegraphics[width=0.678\textwidth]{Figures/Antecedentes_teoricos/Resonancia_alpha_mayor}
			\captionsetup{width=0.8\textwidth}
			\subcaption{}
		\end{subfigure}
	\end{minipage}\begin{minipage}[c]{0.49\textwidth}
		\begin{subfigure}{\textwidth}
			\centering
			\includegraphics[width=0.76878\textwidth]{Figures/Antecedentes_teoricos/Resonancia_alpha_menor1}
			\captionsetup{width=0.8\textwidth}
			\subcaption{}
		\end{subfigure}
	\end{minipage}
	\caption{Soluciones gráficas para la condición resonante para dos casos distintos de la relación de dispersión $\omega\sim k^{\alpha}$.}
	\label{fig:solución_condición_resonante}
\end{figure}

Para las ondas de gravedad $\alpha=1/2$ con lo cual las interacciones resonantes de 3 ondas no están permitidas (sino serían ondas con energía negativa). En el caso particular en que todas las ondas son colineales (reducimos el problema efectivamente a 1D, aunque la energía sigue distribuída en 2D) la mínima cantidad de ondas interactuantes resulta ser $N=5$. \cite{nazarenkoWaveTurbulence2011}

\section{Eliminación de los términos no resonantes.}
Vamos a querer mostrar que efectivamente en el caso de que un término sea no resonante, o sea, que no cumple la condición \eqref{eq:condición_resonancia}, existirá una transformación canónica a nuevas variables $c_{\vec k}, c_{\vec k}^*$ en las cuales ese término no aparece en el Hamiltoniano de interacción. 

A modo ilustrativo \cite{zakharovKolmogorovSpectraTurbulence1992} plantea para una sola variable el Hamiltoniano hasta orden cuatro:

\begin{equation}
	\mathcal{H} = \omega bb^*+\frac{V}{2}(b^2b^*+{b^*}^2) + \frac{U}{6}(b^3+{b^*}^3) + \frac{W}{4}(b^2{b^*}^2) + G(b^3b^*+b{b^*}^3) + R (b^4+{b^*}^4)
\end{equation}

Y pide que la nueva variable sea de la forma (desarrollo perturbativo en $b$):

\begin{equation}
	b = c + A_1c^2+A_2cc^*+A_3{c^*}^2 + B_1c^3+B_2c^*c^2+B_3c{c^*}^2+B_4{c^*}^3+\cdots
\end{equation}

Luego pide que $\{b,b^*\}_{c,c^*}=1$ para que la transformación sea efectivamente canónica y resuelve para los coeficientes $A_i$ y $B_i$ pidiendo que todos los términos no lineales del Hamiltoniano excepto $c^2{c^*}^2$ sean 0. Entonces llega a que:

\begin{equation}
	\mathcal{H} = \omega cc^* + \frac{1}{4} Tc^2{c^*}^2 \qquad T=W-\frac{3V^2}{\omega}-\frac{U^2}{3\omega}
\end{equation}

O sea, que siempre y cuando $\omega \neq 0$ se pueden eliminar los términos no resonantes del Hamiltoniano. 

Es posible hacer exactamente lo mismo para el caso en que no hay triadas resonantes, quedando: 

\begin{equation}
	\mathcal{H} = \frac{1}{4} \int T_{1234} c_1^*c_2^*c_3c_4\delta(\vec k_1+\vec k_2 - \vec k_3 - \vec k_4) d\vec k_1d\vec k_2 d\vec k_3 d\vec k_4
\end{equation}

Donde el coeficiente de interacción resulta:

\begin{equation}
	\begin{split}
		T_{1234} = W_{1234} &-\frac{U_{-1-212}U_{-3-434}}{\omega_3+\omega_4+\omega_{3+4}} + \frac{V^*_{1+212}V_{3+434}}{\omega_1+\omega_2-\omega_{1+2}} 	-\frac{V^*_{131-3}V_{424-2}}{\omega_{4-2}+\omega_2-\omega_4} \\ 
		&- \frac{V^*_{242-4}V_{313-1}}{\omega_{3-1}+\omega_1-\omega_3} 
		-\frac{V^*_{232-3}V_{414-1}}{\omega_{4-1}+\omega_1-\omega_4}-\frac{V^*_{141-4}V_{323-2}}{\omega_{3-2}+\omega_2-\omega_3} 
	\end{split}
\end{equation} 

Notemos que si efectivamente la condición resonante para tres ondas no puede cumplirse (con una relación de dispersión de no decaimiento), entonces los denominadores de $T_{1234}$ no divergen y efectivamente se puede eliminar $\mathcal{H}_3$ del Hamiltoniano de interacción $\mathcal{H}_{int}$.

En general la prohibición de interacciones del tipo $2\rightarrow1$ y $1\rightarrow2$ implican la prohibición de los términos $1\rightarrow3$ y $3\rightarrow1$. Igualmente se eliminan los términos $4\rightarrow0$ y $0\rightarrow4$ del Hamiltoniano, resultando en su versión más simplificada.


Notemos que al interactuar la mitad de las ondas de cada lado ($N/2\rightarrow N/2$) se conserva la cantidad de partículas, asociado a la simetría $U(1)$ que mencionamos antes:

\begin{equation}
	N = \int c^*_{\vec k}c_{\vec k} d\vec k
	\label{eq:Wave_action_integral}
\end{equation}

Que es lo que se conoce como la integral de \textit{Wave action}.


Los distintos términos de $T_{1234}$ se pueden interpretar gráficamente:

\begin{figure}[!ht]
	\centering
	\includegraphics[width=0.7\linewidth]{Figures/Antecedentes_teoricos/Procesos_de_4_ondas}
	\caption{Distintos procesos de cuatro ondas como segundo orden perturbativo a proceso de tres ondas, donde la tercera es virtual y conecta ambos extremos. } % ec
	\label{fig:procesosde4ondas}
\end{figure}


Donde se pueden pensar como un segundo orden de perturbación a un proceso de tres ondas, donde aparece una fuerza virtual que no conserva energía y momento. 



\section{Correspondencia con fluidos con superficie libre.}
Hasta ahora hemos dado un visión general del modelado de un sistema continuo no lineal mediante la teoría de perturbaciones, aplicable a una gran cantidad de sistemas físicos. En particular nos va a interesar mostrar que un fluido con superficie libre que se extiende infinitamente y con profundidad finita se puede describir de esta forma.

Vamos a pensar en un fluido irrotacional e incompresible, de esta forma $\vec u = \vec \nabla \phi$, con $\phi$ el potencial de velocidad. Éste satisface la Ecuación de Laplace.

\begin{equation}
	\nabla^2\phi = 0 \qquad -h<z<\eta
\end{equation}

Donde $-h$ es la posición del fondo y $z=\eta(\vec x)$ es la altura de la superficie libre. Vamos a tener además la condición de contorno en el fondo:

\begin{equation}
	\partial_z\phi=0 \qquad z=-h
\end{equation}

Y las condiciones de contorno cinemática (que sale por continuidad de la velocidad) y dinámica (que sale de Bernoulli por continuidad de presión) respectivamente:

\begin{equation}
	\frac{D\eta}{Dt} = \partial_t \phi + \vec \nabla \phi \cdot \vec \nabla \eta = \partial_z \phi	 \qquad  \partial_t \phi + \frac{1}{2} |\vec \nabla \phi|^2 + g\eta = 0
\end{equation}

Que tendrá solución única dadas la altura de la superficie libre $\eta(\vec x,t)$ y el potencial en la superficie libre $\phi^s=\phi(\vec x, z=\eta(\vec x, t), t)$, que permiten obtener el potencial para todo $z, t$. \cite{meiTheoryApplicationsOcean2016}

Definimos:

\begin{equation}
	w^s = \left(\frac{\partial\phi}{\partial x}\right)_\eta
\end{equation} 

Y transformamos todo a Fourier. La ecuación de Laplace queda:

\begin{equation}
	\frac{\partial^2\phi_k}{\partial z^2} + k^2 \phi_k = 0
\end{equation}

Que podemos resolver por separación de la variable $z$ para que satisfaga la condición de contorno en el fondo:

\begin{equation}
	\phi(\vec k, z, t) = \Phi(\vec k, t) \cosh(k(z+h))
\end{equation}

Luego podemos obtener $\phi^s(\vec x, t)$ al antitransformar. Por otro lado si derivamos y antitransformamos podemos obtener $w^s(\vec x, t)$.

Ahora debemos tomar la aproximación de que la pendiente de la ola (\textit{steepness}) es muy baja, i.e. $k\eta\ll1$. Entonces expandimos $\phi^s$ y $w^s$ en términos del parámetro $k\eta$. Volvemos a transformar y expresamos $\eta$ en términos de su transformada $\eta_k$ para obtener $\phi^s_k$ y $w^s_k$. Por último invertimos las expresiones iterativamente para hallar $\Phi_k$. 

Finalmente podemos definir:
\begin{equation}
	b_k = \left(\frac{g}{2\omega}\right)^{1/2} \eta_k + i \left(\frac{\omega}{2g}\right)^{1/2}\phi^s_k
\end{equation}

Y queda una ecuación de movimiento para la evolución de $b_k$ análoga a las que habíamos encontrado en el caso general. Para los \textit{kernels} de interacción para procesos de tres y cuatro ondas se puede ver el Apéndice de \cite{meiTheoryApplicationsOcean2016}.


\section{Paso a la estadística y la Ecuación Cinética.}
La idea ahora va a ser pasar de la descripción dinámica del sistema a una estadística en función de las funciones de correlación de las amplitudes de ondas.

\subsection*{Resumen Ecuaciones de Movimiento}
En las variables apropiadas los Hamiltonianos sin términos no resonantes nos quedan, para el caso de tres ondas: 


\begin{equation}
	\mathcal{H}_3 = \frac{1}{2}\int \big(V_{123}c_1^*c_2c_3 + \text{c.c.}\big)\delta(\vec k_1-\vec k_2-\vec k_3)d\vec k_1 d\vec k_2 d\vec k_3
\end{equation}

Con su ecuación de movimiento:

\begin{equation}
	i\frac{\partial c_{\vec k}}{\partial t} - \omega_k c_{\vec k} = \int \left(\frac{1}{2}V_{k12}c_1c_2\delta(\vec k - \vec k_1 -\vec k_2) + V^*_{1k2}c_1c_2^*\delta(\vec k_1 -\vec k-\vec k_2)\right)d\vec k_1d\vec k_2
	\label{eq:motion_equation_three_waves}
\end{equation}

Donde primero tenemos $k\rightarrow1+2$ (que es una sola opción) y luego $k+2\rightarrow1$ (que son dos opciones).

Y para el caso de cuatro ondas:

\begin{equation}
	\mathcal{H} = \frac{1}{4}\int T_{1234}c_1^*c_2^*c_3c_4\delta(\vec k_1+\vec k_2-\vec k_3-\vec k_4)d\vec k_1 d\vec k_22 d\vec k_3 d\vec k_4
\end{equation}

Con su ecuación de movimiento:

\begin{equation}
	i\frac{\partial c_{\vec k}}{\partial t} - \omega_k c_{\vec k} = \frac{1}{2} \int T_{k123}c_1^*c_2c_3\delta(\vec k+\vec k_1-\vec k_2-\vec k_2) d\vec k_1 d\vec k_2d\vec k_3
\end{equation}

Que son procesos $2\rightarrow2$.

\subsection*{Paso a la estadística.}
Las ecuaciones de movimiento anteriores son para la evolución temporal de las amplitudes y fases de $c_{\vec k} = |c(\vec k, t)|e^{i\phi(\vec k, t)}$. En el caso en que las nolinealidades son débiles y hay una gran cantidad de modos esta evolución suele ser redundante, ya que incluye la dinámica rápida de $\phi\sim\omega t$ que deja a la evolución lenta de las amplitudes (que es constante en el caso puramente lineal, orden más bajo) virtualmente no afectada. Por esto vamos a querer pasar a una descripción estadística, donde solo trabajaremos con las funciones de correlación para $c_{\vec k}$, donde vamos a promediar en un ensamble.

Vamos a asumir una "caotización" de las fases, de forma tal que aunque al inicio estuvieran correlacionadas, luego de un tiempo ya no lo estarán. Podemos pensar que esto se debe a la dispersión del medio o a que inicialmente están en un equilibrio muy inestable del cual los saca cualquier perturbación del medio. De esta forma, bajo la aproximación de fases aleatorias tenemos las correlaciones:

\begin{equation}
	\begin{split}
		\langle c_{\vec k}\rangle &= \langle |c_{\vec k}|e^{i\phi_{\vec k}}	 \rangle = 0 \\
		\langle c_{\vec k} c_{\pvec k'} \rangle &= \langle |c_{\vec k}||c_{\pvec k'}| e^{i(\phi_{\vec k}+\phi_{\pvec k'})} \rangle = 0 \\
		\langle c_{\vec k} c^*_{\pvec k'} \rangle  &= \langle |c_{\vec k}||c_{\pvec k'}| e^{i(\phi_{\vec k}-\phi_{\pvec k'})} \rangle = n(\vec k) \delta(\vec k - \pvec k')
	\end{split}
\end{equation} 

Acá definimos la acción de onda $n(\vec k) = c_{\vec k}c^*_{\vec k}$, que sería la densidad de "partículas" con número de onda $\vec k$ en la integral \eqref{eq:Wave_action_integral}.  

Por último para los correlatorres de una cantidad impar siempre van a dar 0, y para los de orden 4 y 6 tenemos:

\begin{equation}
	\begin{split}
		\langle	c_1^*c_2^*c_3c_4 \rangle & = n_1 n_2 \big[\delta_{1-3}\delta_{2-4} + \delta_{1-4}\delta_{2-3}\big] \\ 
		\langle c_1^*c_2^*c_3^*c_4c_5c_6 \rangle &= n_1n_2n_3\big[  \delta_{1-4}\delta_{2-5}\delta_{3-6} + \delta_{1-4}\delta_{2-6}\delta_{3-5} + \delta_{1-5}\delta_{2-4}\delta_{3-6} \\ 
		& +\delta_{1-5}\delta_{2-6}\delta_{3-4} + \delta_{1-6}\delta_{2-4}\delta_{3-5} + \delta_{1-6}\delta_{2-5}\delta_{3-6} \big] 
	\end{split}
\end{equation}

Que básicamente las deltas dan que queden dos o tres  $n$'s. Acá es donde producimos la clausura de la jerarquía, al escribir el cuarto momento como un producto de dos segundos momentos (o tres para el sexto). En parte esta aproximación se puede hacer porque las estadísticas son cercanas a la Gaussiana para turbulencia de ondas débil que es esto. \cite{falkovichTurbulence2006} 

\subsection*{Clausura Gaussiana.}
Ya hemos estado asumiendo que el comportamiento estadístico de la turbulencia de ondas débil es cercano al Gaussiano en el tiempo largo, siendo insensible a las condiciones iniciales. \cite{falconLaboratoryExperimentsWave2010} 

Esta propiedad, a la cual nos referimos como clausura Gaussiana ya que nos permite expresar momentos superiores en función del segundo momento y escribir de esta forma una expresión cerrada para la ecuación cinética, viene de una de una aplicación del Teorema Central del Límite. \cite{hasselmannNonlinearEnergyTransfer1962} 


\subsection*{Ecuación cinética de tres ondas.}
Para obtener la ecuación de evolución para $n_{\vec k}$ hay que multiplicar \eqref{eq:motion_equation_three_waves} por $c^*_{\vec k}$y restarle la conjugada por $c_{\pvec k'}$, para finalmente tomar el promedio y reemplazar $\vec k=\pvec k'$:

\begin{equation}
	\frac{\partial n_{\vec k}}{\partial t} = \text{Im} \int\big[V_{k12} \langle c_k^*c_1c_2 \rangle \delta(\vec k- \vec k_1 -\vec k_2) - 2V_{1k2}\langle c_1^*c_kc_2 \rangle \delta(\vec k_1 - \vec k - \vec k_2) \big] d\vec k_1 d\vec k_2
\end{equation} 

Ahora bien $\langle c_1^+c_2c_3 \rangle = 0$ a orden 0. Debemos entonces ir a un orden más en la teoría perturbativa para calcular el correlator de tres puntos. Esto lo hacemos usando la ecuación \eqref{eq:motion_equation_three_waves} por otras dos amplitudes y sumando:

\begin{equation}
	\begin{split}
		\left[i\frac{\partial}{\partial t} + (\omega_1-\omega_2-\omega_3)\right] \langle c_1^*c_2c_3 \rangle &= \int \bigg[-\frac{1}{2}V_{145}^*\langle c_4^*c_5c_2c_3 \rangle \delta(\vec k_1 - \vec k_4 -\vec k_5) + V_{425}^*\langle c_1^*c_5c_3c_4 \rangle \delta(\vec k_4 - \vec k_2 -\vec k_5) \\ 
		&+  V_{435}^*\langle c_1^*c_5c_2c_4 \rangle \delta(\vec k_4 - \vec k_3 -\vec k_5)\bigg] d\vec k_4 d\vec k_5
	\end{split}
\end{equation}

Ahora usamos la expresión a primer orden del correlator de cuatro puntos de antes y llegamos a que:

\begin{equation}
	\left[i\frac{\partial}{\partial t} + (\omega_1-\omega_2-\omega_3)\right] \langle c_1^*c_2c_3 \rangle = V_{123}^*[n_1n_3+n_1n_2-n_2n_3]
\end{equation}

Al segundo orden de perturbaciones esta ecuación pedimos que no dependa del tiempo. Intgrando nos queda un término constanTe $A_{123}/\Delta\omega$ y otro que oscila rápidamente a frecuencia $\Delta\omega t$, que puede ser despreciada al integrar el correlator de tres puntos. De esta forma nos queda esencialmente que:

\begin{equation}
	\langle c_1^*c_2c_3 \rangle = \frac{V_{123}^*(n_1n_2+n_1n_3-n_2n_3)}{\omega_1-\omega_2-\omega_3+i\delta}
\end{equation} 

Se coloca $+i\delta$ para evitar el polo, que vamos a atravesar ya que se da en la condición resonante. Hacemos tender $\delta\rightarrow0$ y usamos que $\text{Im}(1/(\omega+i\delta))\rightarrow-\pi\delta(\omega)$, que es medio residuo al integrar en el eje real y el signo menos sale por la orientación de la curva. \footnote{Queremos la integral de izquierda a derecha en los reales, cerramos por abajo con un semicírculo de forma que encerramos el polo en $\omega=-i\delta$, cuya integral en el infinito dad 0. Luego equivale la integral en reales al medio residuo $-2\pi i/2$ y finalmente tomamos la parte imaginaria.} 


Reemplazando este resultado en la ecuación cinética obtenemos\footnote{En \cite{zakharovKolmogorovSpectraTurbulence1992} p. 67 el signo es $+$, pero por \cite{nazarenkoWaveTurbulence2011} y lo que sigue creo es $-$.}:

\begin{equation}
	\begin{split}
		\frac{\partial n_{\vec k}}{\partial t} &= \pi \int\big[|V_{k12}|^2f_{k12}\delta(\vec k - \vec k_1 - \vec k_2) \delta(\omega - \omega_1 - \omega_2) - 2 |V_{1k2}|^2 f_{1k2} \delta(\vec k_1 - \vec k - \vec k_2) \delta(\omega_1 - \omega - \omega_2) \big]  \\ 
		f_{123} &= n_1n_2n_3\left(\frac{1}{n_1} - \frac{1}{n_2} -  \frac{1}{n_3}\right)
	\end{split}
\end{equation}


\subsection*{Ecuación cinética de cuatro ondas.}
Seguimos un procedimiento análogo al anterior, nuevamente va a dar a primer orden 0, con lo cual hay que ir a un orden más alto y ahora usar el correlator de seis puntos. Finalmente queda:

\begin{equation}
	\begin{split}
		\frac{\partial n_{\vec k}}{\partial t} &= \frac{\pi}{2} \int |T_{k123}|^2 f_{k123} \delta(\vec k + \vec k_1 - \vec k_2 -\vec k_3) \delta(\omega + \omega_1 -\omega_2 - \omega_3) d\vec k_1 d\vec k_2 d\vec k_3 \\
		f_{1234} &= n_1n_2n_3n_4\left(\frac{1}{n_1} + \frac{1}{n_2} - \frac{1}{n_3} - \frac{1}{n_4} \right)
	\end{split}
\end{equation}


\section{Modificaciones a la Ecuación Cinética.}
Se pueden agregar términos de disipación y forzado en la ecuación cinética, mediante una modificación de la ecuación de Hamilton \cite{falkovichTurbulence2006, falconLaboratoryExperimentsWave2010}:

\begin{equation}
	\frac{\partial a_k}{\partial t} = -i\frac{\delta \mathcal{H}}{\delta a_k^*} + f_k(t) - \gamma_{k}a_k
\end{equation}

Donde el primer término dará cuenta de la propagación e interacción de ondas, el segundo del forzado y el tercero de la disipación. Si asumimos que tenemos una fuerza aleatoria, que sea homogénea e isótropa en el espacio y blanca en el tiempo entonces,

\begin{equation}
	\langle f_k(t)f_{k'}(t') \rangle = F(k) \delta(\vec k+ \pvec k')\delta(t'-t)
\end{equation}

Con $F(k)\neq 0$ en el entorno de $k_f$ donde se produce la inyección de energía. Además asumimos $\gamma_k\ll\omega_k$.  El tiempo de disipación se puede estimar como $\tau_{diss}\sim 1/\gamma_k $, que será importante cuando sea del orden de los tiempos no lineales $\tau_{nl}\sim \tau_{diss}$. Para el caso de superficie libre de un fluido tenemos $\gamma_k = 2\nu k^2$. \cite{deikeEtudesExperimentalesNumeriques2013}

La ecuación cinética queda:

\begin{equation}
	\frac{\partial n_k}{\partial t} = F_k - \gamma_k n_k + I_k^3  
\end{equation}

Donde $I^3_k$ es la integral de colisiones para tres ondas en este caso, pero podría ser $I_k^4$ para otro sistema. En \cite{falconLaboratoryExperimentsWave2010} la escriben con una notación alternativa:

\begin{equation}
	\frac{\partial n_{\vec k}}{\partial t} = C_{\vec k} - D_{\vec k} + I_{\vec k}
\end{equation}

Donde $C_{\vec k}$ es la integral de interacciones entre ondas, $D_{\vec k}$ el \textit{damping} y $I_{\vec k}$ la \textit{inyección} de energía.

El caso en que $D_k=I_k=0$ corresponde con el equilibrio termodinámico, o la solución de Rayleigh-Jeans, y está asociada a una equipartición de energía entre todos los modos.


\section*{Propiedades de la Ecuación Cinética.}
\subsection*{Cantidades conservadas}
Como ya se mencionó varias veces el sistema dinámico preserva ciertas cantidades, en particular la energía $E$ y momento $\vec \Pi$ en el caso general, y para el caso de interacciones del tipo $2\rightarrow2$ también $N$.

Podemos ver que estas conservaciones se mantienen en la descripción estadística, a partir de la Ecuación Cinética. En el caso de la energía y momento es intuitivo ya que tenemos las deltas que preservan estas cantidades durante las interacciones. Pensemos como \cite{nazarenkoWaveTurbulence2011} en el caso de una magnitud arbitraria $\Phi$ cuya densidad espectral es $\phi(\vec k) = \rho(\vec k)n(\vec k)$, entonces su evolución temporal será:

\begin{equation}
	\frac{d\Phi}{dt} = \dot \Phi = \frac{d}{dt}\int\phi(\vec k)d\vec k = \int \rho(\vec k)\frac{\partial n(\vec k)}{\partial t}d\vec k
\end{equation}

Ahora podemos reemplazar la derivada temporal de $n_{\vec k}$ por la ecuación cinética. Veamos en particular el caso de tres ondas, el de cuatro es análogo.

\begin{align}
	\frac{\partial n_{\vec k}}{\partial t} &= \int \big(\mathcal{R}_{k12} -\mathcal{R}_{2k1} - \mathcal{R}_{12k}\big) d\vec k_1 d\vec k_2 \\
	\mathcal{R}_{123} &= \pi |V_{123}|^2 n_1n_2n_3\left(\frac{1}{n_1}-\frac{1}{n_2}-\frac{1}{n_3}\right)\delta(\omega_1-\omega_2-\omega_3)\delta(\vec k_1 -\vec k_2-\vec k_3)
\end{align}

Entonces resulta,

\begin{equation}
	\dot \Phi = \int \rho_{\vec k}  \big(\mathcal{R}_{k12} -\mathcal{R}_{2k1} - \mathcal{R}_{12k}\big)    d\vec  k d\vec k_1 d\vec k_2 = \int (\rho_k-\rho_1 -\rho_2) \mathcal{R}_{k12} d\vec k d\vec k_1 d\vec k_2
\end{equation}

Donde en el último igual solo se renombraron las variables de integración cíclicamente (esto se puede hacer ya que estamos integrando en las tres, con lo cual se vuelven mudas).

Podemos ver entonces que si $\rho_k-\rho_1-\rho_2=0$ ocurre en simultáneo con las condiciones de resonancia, entonces la cantidad $\Phi$ se conserva, o sea, $\dot\Phi=0$. Para la energía basta con tomar $\rho_{\vec k}=\omega(\vec k)$, y para la componente $i$-ésima del momento basta con tomar $\rho(\vec k)\vec k_i$. 

En el caso de cuatro ondas por un método análogo nos va a quedar la condición $\rho_k+\rho_1=\rho_2+\rho_3$ y podemos obtener la conservación de la acción de onda tomando simplemente $\rho(\vec k)=1$. 

Una cosa importante a mencionar es que en el caso no degenerado las cantidades $E$, $N$ y $\vec \Pi$ forman un conjunto completo de integrales de movimiento. Pueden aparecer otras constante sólo en casos degenerados \cite{zakharovKolmogorovSpectraTurbulence1992} donde se satisfaga la condición $\rho_k=\rho_1+\rho_2$ para densidades tales que $\rho(\vec k)\neq A\omega(\vec k)+\vec B\cdot\vec k$. De ser el caso decimos que $\omega(\vec k)$ es degenerada.

Un ejemplo de esto es el caso de aguas poco profundas en el modelo de Kadomtsev-Petviashvili, donde se pueden encontrar infinitas integrales de movimiento, propiedad heredada de la ya establecida integrabilidad de la propia ecuación.

\subsection*{Balance de energía.}
Ya establecimos que la energía total se conserva, o sea:

\begin{equation}
	\frac{dE}{dt} = \int \frac{\partial E_{\vec k}}{\partial t} d\vec k =\int \omega \frac{\partial n_{\vec k}}{\partial t} d\vec k  
\end{equation}

Ahora podemos usar nuevamente la ecuación cinética nuevamente, ahora escita de la forma: $\partial_t n_{\vec k} = I_{\vec k}^N$, siendo $I_k^N$ las integrales de colisiones. Esto va a ser cierto en el rango inercial (siempre que la escala de inyección y disipación estén lo suficientemente separadas). Entonces podemos escribir una ecuación de continuidad para la energía

%\begin{equation}
%	\varepsilon \equiv \frac{\partial }{\partial t} \int n_{\vec k} \omega d\vec k 
%\end{equation}
 
\begin{equation}
	\frac{\partial E(\vec k)}{\partial t} + \vec\nabla_{\vec k} \cdot \vec \varepsilon = 0  \qquad \vec\nabla_{\vec k} \cdot \vec \varepsilon \equiv - \omega(\vec k)\;I_{\vec k}^N
	\label{eq:energy_balance}
\end{equation}


Donde definimos el flujo de energía $\vec \varepsilon$. O su versión 1D con la normalización de cilíndricas:

\begin{equation}
	\frac{\partial E(k)}{\partial t} + \frac{\partial \varepsilon}{\partial k} = 0
\end{equation}

Notamos que ecuaciones estacionarias de la ecuacion cinética ($\partial_tn_k = 0$) implican que las integrales de colisiones son 0, y se corrseponden con un flujo constante de energía $\varepsilon$ entre escalas. Para la relación de dispersión $\omega\sim k^{alpha}$ son los espectros de KZF, análogos a la Ley de -5/3 (c.f. \eqref{eq:Kolmogorov_5_3}) de Kolmogorov para turbulencia normal.

Lo mismo se puede hacer para el momento:

\begin{equation}
	\frac{\partial\vec \pi(\vec k)}{\partial t} + \vec\nabla_{\vec k} \cdot  \mathbf{R} = 0  \qquad \vec\nabla_{\vec k} \cdot  \mathbf{R} \equiv - \vec k\;I_{\vec k}^N 
\end{equation}

Donde $\mathbf{R}$ es un tensor de rango 2.

Para el caso en que se conserva la integral de acción de onda $N=\int n_k dk$ vamos a definir el flujo de \textit{wave action} como:

\begin{equation}
	\frac{\partial n(\vec k)}{\partial t} + \vec\nabla_{\vec k} \cdot \vec \zeta = 0  \qquad \vec\nabla_{\vec k} \cdot \vec \zeta \equiv - \;I_{\vec k}^N
\end{equation}

O su versión en 1D:

\begin{equation}
	\frac{\partial n_k}{\partial t} + \frac{\partial \zeta}{\partial k} = 0
\end{equation}


\subsection*{Cuando hay disipación.}
Cuando agregamos el término disipativo en la ecuación cinética $+\Gamma({\vec k})n_{\vec k}$ entonces en el caso estacionario nos va a quedar:

\begin{equation}
	I^N(\vec k) + \Gamma(\vec k)n(\vec k) = 0
\end{equation}

Esto también modela la inyección de ondas para $\Gamma(\vec k)>0$. 

Sabemos por lo que mostramos antes que vale por las deltas de la integral de colisiones:

\begin{equation}
	\int I^N({\vec k}) \; \omega d\vec k = \int I^N({\vec k})\; \vec k d\vec k = 0 
\end{equation}

Entonces por la conservación de la energía y el momento tenemos, de multiplicar por $\omega$ y $\vec k$ la ecuación cinética estacionaria e integrar obtenemos:

\begin{equation}
	\int \Gamma(\vec k) \omega n(\vec k)d\vec k = 0 \qquad \int \Gamma(\vec k) \vec k n(\vec k)d\vec k = 0
\end{equation}


Finalmente si integramos la ecuación de continuidad la parte de la integral de colisiones va a dar 0, con lo cual efectivamente nos va a quedar que la variación del flujo es:

\begin{equation}
	\frac{d\varepsilon}{dk} = \Gamma(\vec k)E(\vec k)
\end{equation}

Y si vemos lo que pasa en una esfera se puede mostrar que el flujo de energía por la superficie se corresponde con el disipado y el inyectado. \cite{falkovichTurbulence2006}

\section{Espectro de energía y PSD.}
Nos interesará estudiar la transferencia de energía entre escalas experimentalmente, que estará caracterizada por la densidad espectral de energía $E_k$ tal que la magnitud $E = \int E_k dk$ se conserva. 

Para esto vamos a empezar dando las definiciones básicas de lo que estaremos usando.

Como mencioné anteriormente, la energía total va a venir dada por una distribución, que puede escribirse ya sea en el espacio de $k$ o de $\omega$, vinculados ambos a través de la relación de dispersión lineal $\omega(k)$. Entonces:

\begin{equation}
	E = \int E(\vec k) d\vec k = \int k E(\vec k) dkd\theta \equiv \int E(k) dk = \int E(\omega) d\omega
\end{equation}

O sea, estamos definiendo $E(k) = 2\pi E(\vec k)$ y la relación entre las densidades en espacio de frecuencia y momento está dada por: $E(k) dk = E(\omega) d\omega$.

Esta energía será la de la superficie libre. En principio sabemos que se dará un balance entre la energía cinética y la potencial, como una especie de equipartición \cite{kunduFluidMechanics2014}. Se suele calcular la potencial (a partir de las mediciones para la superficie libre $\eta$) y se multiplica por 2 \cite{deikeEtudesExperimentalesNumeriques2013}. Vamos a entonces trabajar con las energías potenciales, cuya expresión depende de si tenemos ondas de gravedad o capilares. Tenemos que:

\begin{equation}
	E_p^g = \frac{1}{2} \rho g \int \eta^2 dS \qquad E_p^s = \gamma \int\sqrt{1+(\nabla\eta)^2} - 1 dS
\end{equation}

Acá la energía por capilaridad es básicamente la integral del elemento de superficie $ds$. Se puede aproximar a primer orden para una onda armónica en el caso lineal (con $eta\ll1$), entonces \cite{deikeEtudesExperimentalesNumeriques2013}

\begin{equation}
	E_p^s = \frac{1}{2} \gamma\int k^2\eta^2dS
\end{equation}

Usando que $(1+\varepsilon)^n\approx1+n\varepsilon$. Ahora vamos a querer pasar al espacio de Fourier, transformando las energías:

\begin{equation}
	E_p^g = \frac{\rho g}{2} \int |\eta_{\vec k}|^2d\vec k \qquad E_p^s = \frac{\gamma}{2} \int k^2|\eta_{\vec k}|^2d\vec k
\end{equation}

Donde $\eta_{\vec k}$ es la transformada de Fourier de $\eta$:

\begin{equation}
	\eta(\vec k) = \frac{1}{2\pi}\int\eta(\vec r)e^{-i\vec k\cdot\vec r} d\vec r
\end{equation}

Podemos integrar en cilíndricas para definir $\eta_k = 2\pi k\eta_{\vec k}$. Las densidades espectrales de energía van a ser los integrandos. 

Por último vamos a definir la Power Spectral Density (PSD) que va a ser lo más sencillo para trabajar experimentalmente:

\begin{equation}
	S_\eta(\omega) = \frac{1}{T} |\eta_\omega|^2 \qquad S_\eta(k) = \frac{1}{L^2} |\eta_k|^2
\end{equation}

Acá ahora $\eta(\omega) = \frac{1}{\sqrt{2\pi}}\int\eta(t)e^{i\omega t}$ es la transformada de Fourier respecto al tiempo. Además para obtener ya sea $\eta(t)$ o $\eta(\vec r)$ se puede promediar en el tiempo o espacio, con lo cual en realidad sería $S_\eta\propto\langle |\eta|^2 \rangle$ (y técnicamente el promedio sería en el ensamble, pero consideramos al sistema ergódico). \cite{falconExperimentsSurfaceGravity2022}

Podemos relacionar las PSD en ambos espacios como $S(\omega)d\omega=S(k)dk$. De esta forma entonces podemos obtener la densidad de energía a partir de la PSD como:

\begin{equation}
		E^g(k) = \frac{\rho g}{2} S_\eta(k) \qquad E^s(k)=\frac{\gamma}{2}k^2S_\eta(k)
\end{equation}


Una cosa interesante es que si escribimos explícitamente $S_\eta(k) \propto k|\eta_{\vec k}|^2$ entonces nos queda en ambos casos:

\begin{equation}
	E(k) \propto \omega^2|\eta_{\vec k}|^2 \propto |\dot \eta_{\vec k}|^2 
\end{equation}

Que sería la energía cinética efectivamente.


\section{Espectro de Kolmogorov-Zakharov-Filonenko.}
Experimentalmente es posible observar que el espectro de energía para la elevación de la superficie libre en un determinado rengo de números de ondas (de equilibrio) resulta $E_\omega  \sim A\omega^s$. En este rango la mayor contribución a la dinámica se debe a los efectos no lineales, ya que los viscosos son menores. Es posible calcular los exponentes de forma analítica a partir de las ecuaciones para la superficie libre \cite{zakharovEnergySpectrumStochastic1967}

La densidad de energía se puede escribir como:

\begin{equation}
	E(\vec k) = \omega(\vec k) n(\vec k)
\end{equation}  

Donde $n$ es la \textit{wave action}, análoga a la cantidad de partículas (o ondas) con ese número de onda.

\subsection*{Ondas de gravedad \cite{zakharovEnergySpectrumStochastic1967}}
Tenemos que $E(\omega)=\omega^4n(\omega)$ y pueden calcular que para $n(\omega)=A\omega^s$ hay soluciones $s=-1$, que se corresponde con la distribución de Rayleigh-Jeans (equilibrio termodinámico para el caso lineal, con PDF Gaussiana \cite{nazarenkoWaveTurbulence2011}), y $s=-8$ que es la análoga al espectro de Kolmogorov para turbulencia hidrodinámica.

Las equivalencias son \cite{falconExperimentsSurfaceGravity2022}:

\begin{equation}
	E_k^g \sim P^{1/3} g^{1/2} k^{-5/2} \qquad S_k^g \sim P^{1/3} g^{-1/2} k^{-5/2} \qquad S_\omega^g  \sim P^{1/3} g \omega^{-4}
\end{equation}  

Es posible llegar a estos resultados de forma dimensional, pero para que el exponente sea único hay que asumir a priori que el flujo en el espectro de energía va en el orden $P^{1/(N-1)}$ con $N$ el número de ondas interactuantes más bajo, siendo $N=4$ para ondas de gravedad.

Para hacer los cambios hay que usar la relación de dispersión:

\begin{equation}
	\omega^2=gk
\end{equation}

\subsection*{Ondas de capilaridad \cite{zakharovWeakTurbulenceCapillary1971}}
Al igual que para las ondas de gravedad calculan $n(\omega)$, llegando esta vez a los resultados de $s=-1$ que es la de Rayleigh-Jeans y $s=-17/6$ que es la que nos interesa. Con esto:

\begin{equation}
	E_k^s \sim P^{1/2} \left(\frac{\gamma}{\rho}\right)^{1/4} k^{-7/4} \qquad S_k^s \sim P^{1/2} \left(\frac{\gamma}{\rho}\right)^{-3/4} k^{-15/4} \qquad S_\omega^s  \sim P^{1/2} \left(\frac{\gamma}{\rho}\right)^{1/6} \omega^{-17/6}
\end{equation}   

Acá la relación de dispersión es:

\begin{equation}
	\omega^2 = \frac{\gamma}{\rho}k^3
\end{equation}


\subsection*{Deducción dimensional.}
Para que sea única dijimos que hay que asumir cómo escala el espectro de energía respecto a la cantidad de ondas interactuantes. Esto surge del balance de energía \eqref{eq:energy_balance}:

\begin{equation}
	\varepsilon \sim \frac{\partial E_k}{\partial t} dk \sim \omega \frac{\partial n_k}{\partial t} dk \sim I_k^N \sim n_k^{N-1}
\end{equation}

Por la ecuación cinética. Por lo tanto: \cite{deikeEtudesExperimentalesNumeriques2013}

\begin{equation}
	E_k \sim n_k \sim \varepsilon^{1/(N-1)}
\end{equation}



\section{Doble cascada para $N$ par} % a sub 
A partir de los resultados para los espectros KZF se puede ver que existe una cascada directa de energía, sin embargo, para el caso en que las interacciones se dan entre un número par de ondas, tenemos otro invariante además de la energía, que es la \textit{wave action}, que tendrá una cascada inversa. \cite{nazarenkoWaveTurbulence2011}

Para el caso de ondas de gravedad, con interacciones de 4 ondas, tenemos el espectro:

\begin{equation}
	E^g_k \sim g^{2/3} \zeta^{1/3} k^{-7/3}
\end{equation}

Con $\zeta$ el flujo de \textit{wave action} (ratio de disipación)



\section{Turbulencia de Ondas Discreta}
Ahora van a estar discretizadas las frecuencias posibles del recinto, de forma tal que entre un número entero de longitudes de onda en el mismo. Por lo tanto la cantidad de interacciones posibles entre ondas se va a ver limitada. 

\subsection*{Quasi-resonancias.}

\subsection*{Clusters de interacción.}

\subsection*{Modelo de pila de arena.}














%Estos son los antecedentes Teóricos. Más antecedentes.
%
%Esta es una ecuación:
%
%\begin{equation}
%	A = B + \phi
%	\label{eq:eq_1}
%\end{equation}
%
%Y esta es u referencia a la ecuación \eqref{eq:eq_1}. 
%
%Y por último una prueba del auto-guardado en Github. Y una más. 
%Y esta sería la última. 

% IMÁGENES Y TABLAS
%
% A la izquierda 
% \begin{figure}[!ht]
%	     \begin{minipage}[c]{0.5\textwidth}
%		         \centerfloat
%		         \includegraphics[width=0.8\textwidth]{}
%		     \end{minipage}
%	         \begin{minipage}[c]{0.49\textwidth}
%		         \captionsetup{width=\textwidth}  
%		         \caption{.}
%		     \label{fig:}
%		     \end{minipage}
%	 \end{figure}
%
%A la derecha
% \begin{figure}[!ht]
%	     \begin{minipage}[c]{0.5\textwidth}
%		           \captionsetup{width=\textwidth}  
%		           \caption{.}
%		           \label{fig:}
%		     \end{minipage}
%	     \begin{minipage}[c]{0.49\textwidth}
%		         \centerfloat
%		         \includegraphics[width=0.8\textwidth]{}
%		     \end{minipage}
%	 \end{figure}
% 
% Ancho completo
% \begin{figure}[!ht]
%	     \centerfloat
%	     \includegraphics[width=0.9\textwidth]{}
%	     \caption{}
%	     \label{fig:}
%	 \end{figure}
%
% Wrapfigure
% \begin{wrapfigure}{r}{0.5\textwidth}
%	         \vspace{-20pt}
%	         \centerfloat
%	         \includesvg[width=0.3\textwidth]{}
%	         \captionsetup{width=0.45\textwidth}
%	         \caption{.}
%	         \label{fig:}
%	 \end{wrapfigure}
%
% Generador de tabals: https://www.tablesgenerator.com/
% Tablas con líneas más gruesas: https://tex.stackexchange.com/questions/41758/how-can-i-reproduce-this-table-with-thick-lines
% 
% Tabla y figura (https://tex.stackexchange.com/questions/417505/table-just-below-a-figure)
% \begin{minipage}{0.99\textwidth}
%	   \begin{minipage}[c]{0.49\textwidth}
%		         \centerfloat
%		         \includegraphics[width=\textwidth]{}
%		         \centering
%		         \captionsetup{width=.8\linewidth}
%		         \captionof{figure}{.}
%		         \label{fig:}
%		   \end{minipage}
%	   \hfill
%	   \begin{minipage}[c]{0.49\textwidth}
%		         \centerfloat
%		         \includegraphics[width=\textwidth]{}
%		         \centering
%		         \captionsetup{width=.8\linewidth}
%		         \captionof{table}{.}
%		          \label{tab:}
%		     \end{minipage}
%	 \end{minipage}
% 
% Subfiguras
%\begin{figure}[!ht]
%	\begin{minipage}[c]{0.5\textwidth}
%		\begin{subfigure}{\textwidth}
%			\centering
%			\includegraphics[width=0.8\textwidth]{}
%			\captionsetup{width=0.8\textwidth}
%			\subcaption{.}
%			\label{fig:}
%		\end{subfigure}
%	\end{minipage}\begin{minipage}[c]{0.49\textwidth}
%		\begin{subfigure}{\textwidth}
%			\centering
%			\includegraphics[width=0.8\textwidth]{}
%			\captionsetup{width=0.8\textwidth}
%			\subcaption{.}
%			\label{fig:}
%		\end{subfigure}
%	\end{minipage}
%	\caption{}
%	\label{fig:}
%\end{figure}
% 
% Tablas/figuras iguales side by side
% \begin{table}[]
%	     \parbox{0.5\textwidth}{
%		     \centering
%		     \includegraphics[width=0.95\linewidth]{}
%		     \caption{}
%		     \label{tab:}
%		 }
%	 \parbox{0.5\textwidth}{
%		     \centering
%		     \includegraphics[width=0.9\linewidth]{}
%		     \caption{}
%		     \label{tab:}
%		 }
%	 \end{table}
% 
% Tablas ancho completo
% \begin{table}[!h]
%	     \centerfloat
%	     \caption{.}
%	     \label{tab:}
%	 \end{table}