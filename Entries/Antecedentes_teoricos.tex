\entry{Antecedentes Teóricos}

\section{Espectro de energía y PSD.}
Nos interesará estudiar la transferencia de energía entre escalas experimentalmente, que estará caracterizada por la densidad espectral de energía $E_k$ tal que la magnitud $E = \int E_k dk$ se conserva. 

Para esto vamos a empezar dando las definiciones básicas de lo que estaremos usando.

Como mencioné anteriormente, la energía total va a venir dada por una distribución, que puede escribirse ya sea en el espacio de $k$ o de $\omega$, vinculados ambos a través de la relación de dispersión lineal $\omega(k)$. Entonces:

\begin{equation}
	E = \int E(\vec k) d\vec k = \int k E(\vec k) dkd\theta \equiv \int E(k) dk = \int E(\omega) d\omega
\end{equation}

O sea, estamos definiendo $E(k) = 2\pi E(\vec k)$ y la relación entre las densidades en espacio de frecuencia y momento está dada por: $E(k) dk = E(\omega) d\omega$.

Esta energía será la de la superficie libre. En principio sabemos que se dará un balance entre la energía cinética y la potencial, como una especie de equipartición \cite{kunduFluidMechanics2014}. Se suele calcular la potencial (a partir de las mediciones para la superficie libre $\eta$) y se multiplica por 2 \cite{deikeEtudesExperimentalesNumeriques2013}. Vamos a entonces trabajar con las energías potenciales, cuya expresión depende de si tenemos ondas de gravedad o capilares. Tenemos que:

\begin{equation}
	E_p^g = \frac{1}{2} \rho g \int \eta^2 dS \qquad E_p^s = \gamma \int\sqrt{1+(\nabla\eta)^2} - 1 dS
\end{equation}

Acá la energía por capilaridad es básicamente la integral del elemento de superficie $ds$. Se puede aproximar a primer orden para una onda armónica en el caso lineal (con $eta\ll1$), entonces \cite{deikeEtudesExperimentalesNumeriques2013}

\begin{equation}
	E_p^s = \frac{1}{2} \gamma\int k^2\eta^2dS
\end{equation}

Usando que $(1+\varepsilon)^n\approx1+n\varepsilon$. Ahora vamos a querer pasar al espacio de Fourier, transformando las energías:

\begin{equation}
	E_p^g = \frac{\rho g}{2} \int |\eta_{\vec k}|^2d\vec k \qquad E_p^s = \frac{\gamma}{2} \int k^2|\eta_{\vec k}|^2d\vec k
\end{equation}

Donde $\eta_{\vec k}$ es la transformada de Fourier de $\eta$:

\begin{equation}
	\eta(\vec k) = \frac{1}{2\pi}\int\eta(\vec r)e^{-i\vec k\cdot\vec r} d\vec r
\end{equation}

Podemos integrar en cilíndricas para definir $\eta_k = 2\pi k\eta_{\vec k}$. Las densidades espectrales de energía van a ser los integrandos. 

Por último vamos a definir la Power Spectral Density (PSD) que va a ser lo más sencillo para trabajar experimentalmente:

\begin{equation}
	S_\eta(\omega) = \frac{1}{T} |\eta_\omega|^2 \qquad S_\eta(k) = \frac{1}{L^2} |\eta_k|^2
\end{equation}

Acá ahora $\eta(\omega) = \frac{1}{\sqrt{2\pi}}\int\eta(t)e^{i\omega t}$ es la transformada de Fourier respecto al tiempo. Además para obtener ya sea $\eta(t)$ o $\eta(\vec r)$ se puede promediar en el tiempo o espacio, con lo cual en realidad sería $S_\eta\propto\langle |\eta|^2 \rangle$ (y técnicamente el promedio sería en el ensamble, pero consideramos al sistema ergódico). \cite{falconExperimentsSurfaceGravity2022}

Podemos relacionar las PSD en ambos espacios como $S(\omega)d\omega=S(k)dk$. De esta forma entonces podemos obtener la densidad de energía a partir de la PSD como:

\begin{equation}
		E^g(k) = \frac{\rho g}{2} S_\eta(k) \qquad E^s(k)=\frac{\gamma}{2}k^2S_\eta(k)
\end{equation}


Una cosa interesante es que si escribimos explícitamente $S_\eta(k) \propto k|\eta_{\vec k}|^2$ entonces nos queda en ambos casos:

\begin{equation}
	E(k) \propto \omega^2|\eta_{\vec k}|^2 \propto |\dot \eta_{\vec k}|^2 
\end{equation}

Que sería la energía cinética efectivamente.


\section{Espectro de Kolmogorov-Zakharov-Filonenko.}
Experimentalmente es posible observar que el espectro de energía para la elevación de la superficie libre en un determinado rengo de números de ondas (de equilibrio) resulta $E_\omega  \sim A\omega^s$. En este rango la mayor contribución a la dinámica se debe a los efectos no lineales, ya que los viscosos son menores. Es posible calcular los exponentes de forma analítica a partir de las ecuaciones para la superficie libre \cite{zakharovEnergySpectrumStochastic1967}

La densidad de energía se puede escribir como:

\begin{equation}
	E(\vec k) = \omega(\vec k) n(\vec k)
\end{equation}  

Donde $n$ es la \textit{wave action}, análoga a la cantidad de partículas (o ondas) con ese número de onda.

\subsection*{Ondas de gravedad \cite{zakharovEnergySpectrumStochastic1967}}



\subsection*{Ondas de capilaridad \cite{zakharovWeakTurbulenceCapillary1971}}





%Estos son los antecedentes Teóricos. Más antecedentes.
%
%Esta es una ecuación:
%
%\begin{equation}
%	A = B + \phi
%	\label{eq:eq_1}
%\end{equation}
%
%Y esta es u referencia a la ecuación \eqref{eq:eq_1}. 
%
%Y por último una prueba del auto-guardado en Github. Y una más. 
%Y esta sería la última. 

% IMÁGENES Y TABLAS
%
% A la izquierda 
% \begin{figure}[!ht]
%	     \begin{minipage}[c]{0.5\textwidth}
%		         \centerfloat
%		         \includegraphics[width=0.8\textwidth]{}
%		     \end{minipage}
%	         \begin{minipage}[c]{0.49\textwidth}
%		         \captionsetup{width=\textwidth}  
%		         \caption{.}
%		     \label{fig:}
%		     \end{minipage}
%	 \end{figure}
%
%A la derecha
% \begin{figure}[!ht]
%	     \begin{minipage}[c]{0.5\textwidth}
%		           \captionsetup{width=\textwidth}  
%		           \caption{.}
%		           \label{fig:}
%		     \end{minipage}
%	     \begin{minipage}[c]{0.49\textwidth}
%		         \centerfloat
%		         \includegraphics[width=0.8\textwidth]{}
%		     \end{minipage}
%	 \end{figure}
% 
% Ancho completo
% \begin{figure}[!ht]
%	     \centerfloat
%	     \includegraphics[width=0.9\textwidth]{}
%	     \caption{}
%	     \label{fig:}
%	 \end{figure}
%
% Wrapfigure
% \begin{wrapfigure}{r}{0.5\textwidth}
%	         \vspace{-20pt}
%	         \centerfloat
%	         \includesvg[width=0.3\textwidth]{}
%	         \captionsetup{width=0.45\textwidth}
%	         \caption{.}
%	         \label{fig:}
%	 \end{wrapfigure}
%
% Generador de tabals: https://www.tablesgenerator.com/
% Tablas con líneas más gruesas: https://tex.stackexchange.com/questions/41758/how-can-i-reproduce-this-table-with-thick-lines
% 
% Tabla y figura (https://tex.stackexchange.com/questions/417505/table-just-below-a-figure)
% \begin{minipage}{0.99\textwidth}
%	   \begin{minipage}[c]{0.49\textwidth}
%		         \centerfloat
%		         \includegraphics[width=\textwidth]{}
%		         \centering
%		         \captionsetup{width=.8\linewidth}
%		         \captionof{figure}{.}
%		         \label{fig:}
%		   \end{minipage}
%	   \hfill
%	   \begin{minipage}[c]{0.49\textwidth}
%		         \centerfloat
%		         \includegraphics[width=\textwidth]{}
%		         \centering
%		         \captionsetup{width=.8\linewidth}
%		         \captionof{table}{.}
%		          \label{tab:}
%		     \end{minipage}
%	 \end{minipage}
% 
% Subfiguras
%\begin{figure}[!ht]
%	\begin{minipage}[c]{0.5\textwidth}
%		\begin{subfigure}{\textwidth}
%			\centering
%			\includegraphics[width=0.8\textwidth]{}
%			\captionsetup{width=0.8\textwidth}
%			\subcaption{.}
%			\label{fig:}
%		\end{subfigure}
%	\end{minipage}\begin{minipage}[c]{0.49\textwidth}
%		\begin{subfigure}{\textwidth}
%			\centering
%			\includegraphics[width=0.8\textwidth]{}
%			\captionsetup{width=0.8\textwidth}
%			\subcaption{.}
%			\label{fig:}
%		\end{subfigure}
%	\end{minipage}
%	\caption{}
%	\label{fig:}
%\end{figure}
% 
% Tablas/figuras iguales side by side
% \begin{table}[]
%	     \parbox{0.5\textwidth}{
%		     \centering
%		     \includegraphics[width=0.95\linewidth]{}
%		     \caption{}
%		     \label{tab:}
%		 }
%	 \parbox{0.5\textwidth}{
%		     \centering
%		     \includegraphics[width=0.9\linewidth]{}
%		     \caption{}
%		     \label{tab:}
%		 }
%	 \end{table}
% 
% Tablas ancho completo
% \begin{table}[!h]
%	     \centerfloat
%	     \caption{.}
%	     \label{tab:}
%	 \end{table}